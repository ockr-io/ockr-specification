\documentclass{article}
\usepackage{subfiles}
\usepackage{float}
\usepackage{titlesec}
\usepackage{listings}
\usepackage{color}
\usepackage{qrcode}
\usepackage{algorithm}
\usepackage{algpseudocode}

\title{Ockr Specification}

\author{
  Bormann, Fabian\\
  \texttt{fabian.bormann@ockr.io}
  \and
  \qrcode{https://github.com/ockr-io/ockr-specification/graphs/contributors}
}
\date{\today}

\definecolor{dkgreen}{rgb}{0,0.6,0}
\definecolor{gray}{rgb}{0.5,0.5,0.5}
\definecolor{mauve}{rgb}{0.58,0,0.82}

\lstset{frame=tb,
  language=Java,
  aboveskip=3mm,
  belowskip=3mm,
  showstringspaces=false,
  columns=flexible,
  basicstyle={\small\ttfamily},
  numbers=none,
  numberstyle=\tiny\color{gray},
  keywordstyle=\color{blue},
  commentstyle=\color{dkgreen},
  stringstyle=\color{mauve},
  breaklines=true,
  breakatwhitespace=true,
  tabsize=3
}

\begin{document}

\maketitle

\tableofcontents
\newpage

\section{Abstract}

Ockr aims to set a standard for creating machine-readable and reliable documents, 
enabling the verification of their authenticity. In short Ockr tries to tackle 
three problems:

\begin{enumerate}
    \item How can we make sure that a document is machine-readable?
    \item How can we ensure that a document has not been modified?
    \item How can we ensure that a document really has been issued by a certain authority?
\end{enumerate}
\newpage

\subfile{pages/motivation}
\subfile{pages/process}
\subfile{pages/algorithms}
\subfile{pages/implementation}

\end{document}